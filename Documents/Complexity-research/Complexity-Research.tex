\documentclass[conference]{IEEEtran}%\documentclass[]{IEEEconf}%In case issues arise with the current documentclass

%opening
\title{Complexity}
\author{\IEEEauthorblockN{jok13}}

\begin{document}
	\maketitle
	\nocite{*}% Added by me
	\IEEEpeerreviewmaketitle
	
	%Look at different graphs; find which algoriths work well(?)
	%Different graph-complexity
	%Algorithm-complexity - O(n), P/NP, etc.

\section{Informed vs Uninformed Search}
A* has lower complexity if it prunes (ignores some nodes - e.g those with a higher total path-cost). I believe this is exactly what any correct implementation of A* would do, as it stops searching/expanding nodes as soon as the goal node is at the front of the priority queue (Not necessarily when it is first discovered - e.g when it is possible to achieve a lower path-cost via another node). Any nodes still in the queue or that have not been expanded yet can be safely ignored, as there cannot exist any route with a lower total path-cost (given that all intermediate path-costs are positive) as soon as the goal node is at the front of the queue.

\section{Sort}
InsertionSort: O(nk), where n is the number of elements to insert into the list (eg. children into queue), and k is the position the child is to be inserted into.

Binary: O(k log n) eliminate half of the list at each step. n is the number of items in the sorted list, and k is the number of items to insert.

Example: Sort 1-10 elements into a list with 1024 items. Correct position is 1;10; 11; 100.
\begin{tabular}{|r|c|c|c|c|c|}
	\hline 
	\rule[-1ex]{0pt}{2.5ex}
	&InsertionSort &&&& Binary  \\ 
	\hline 
	\rule[-1ex]{0pt}{2.5ex} 1&1 & 10 && 10 & 100 \\ 
	\hline 
	\rule[-1ex]{0pt}{2.5ex} 10&10 & 100 && 10 & 100 \\ 
	\hline 
	\rule[-1ex]{0pt}{2.5ex} 11&11 & 110 && 10 & 100 \\ 
	\hline 
	\rule[-1ex]{0pt}{2.5ex} 100&100 & 1000 && 10 & 100 \\ 
	\hline 
\end{tabular}\\

Quicksort: O(n log n) (I think)

\section{Storage}
Empty String: ~40Bytes\\
Pointer/Reference: 32-64bit (depending on processor)

I'm guessing pointers are much better to pass around than Strings.


\section{Complexity}
%TODO Russel & Norvig - P82 - Sec: 3.2 (Complexity)
%TODO Russel & Norvig - P1037 - Mathematical Background(APPENDIX) (Complexity, O(), and P/NP Notation)
 $O(log\, n):\\O(log\, 8)=3\\2^3=8\\O(log\, 1024)=10\\1024=2^{10}$

\bibliographystyle{IEEEtran}%!!!had a % before it!!! add it again if problems arise!!!
%\bibliography{references}

\end{document}
