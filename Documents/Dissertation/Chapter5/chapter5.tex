\chapter{Evaluation}

%Examiners expect to find in your dissertation a section addressing such questions as:

%\begin{itemize}
%   \item Were the requirements correctly identified? 
%   \item Were the design decisions correct?
%   \item Could a more suitable set of tools have been chosen?
%   \item How well did the software meet the needs of those who were expecting to use it?
%   \item How well were any other project aims achieved?
%   \item If you were starting again, what would you do differently?
%\end{itemize}

%Such material is regarded as an important part of the dissertation; it should demonstrate that you are capable not only of carrying out a piece of work but also of thinking critically about how you did it and how you might have done it better. This is seen as an important part of an honours degree. 

%There will be good things and room for improvement with any project. As you write this section, identify and discuss the parts of the work that went well and also consider ways in which the work could be improved. 

%Review the discussion on the Evaluation section from the lectures. A recording is available on Blackboard.
\section{Completed work}
\textbf{NOTES: - DELETE THIS -}
\begin{itemize}
	\item Occasional NullPointerExceptions (and a few other exceptions) when running the system. Problem originates in Mapsforge's code, not mine. Example: unable to load map-tile...
	\item Project started a month later than expected; unable to find supervisor(s) - only two (three?) potential supervisors replied to emails.
	\item No contact with supervisor for a month.
	\item Routes not always correct; caused by incorrect labelling in the OSM database. (Show Llandinam as example?).
\end{itemize}



\section{Future work}
%Future work detailing your ideas about the requirements you didn't have time for.
\textbf{NOTES: - DELETE THIS -}
\begin{itemize}
	\item Display route-distance on map and improve the GUI.
	\item Automatic retrieval of Nodes and Ways from OSM.%Issue #25 & #23 (& #10?) - Github
	\item Store data in better data-structures for faster indexing and lower memory-requirements.
	\subitem Hash table/map?
	\subitem Eliminate the need to store Ways by filtering out inaccessible Ways, and then storing Node-connections within the Nodes themselves? Way-type might not be needed after the filtering.
	\item Let user choose their disability/method of locomotion - eg. Manual/Motorised wheelchair, crutches, foot, bike, etc.
	\item Implement more/better search-algorithms.
	\item Update/add data to the OpenStreetMap databases to improve the usefulness and accuracy of my system's suggestions.
	\subitem Aberystwyth University should be able to provide records of accessible entrances, wheelchair-ramps, etc. on campus. That might be a good place to start.
	\item Implement localisation (eg. GPS) to make it easier for people to use the system.
	\item Add search-functionality to let users search for buildings and places rather than forcing them to click on the map; the user might know what the place they want to go to is called, but not necessarily where it is located.
	\item Include the data-type \textquotedblleft Relations\textquotedblright in route-planning; especially bus-routes.
	\item Use services like \cite{geofabrik,osrm} to download larger map-areas than what is possible using the export-option on openstreetmap.org (Which is what I currently do).
\end{itemize}