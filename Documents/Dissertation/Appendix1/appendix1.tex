\chapter{Third-Party Code and Libraries}

%If you have made use of any third party code or software libraries, i.e. any code that you have not designed and written yourself, then you must include this appendix. 

%As has been said in lectures, it is acceptable and likely that you will make use of third-party code and software libraries. The key requirement is that we understand what is your original work and what work is based on that of other people. 

%Therefore, you need to clearly state what you have used and where the original material can be found. Also, if you have made any changes to the original versions, you must explain what you have changed. 

%As an example, you might include a definition such as: 

%Apache POI library � The project has been used to read and write Microsoft Excel files (XLS) as part of the interaction with the client�s existing system for processing data. Version 3.10-FINAL was used. The library is open source and it is available from the Apache Software Foundation 
%\cite{apache_poi}. The library is released using the Apache License 
%\cite{apache_license}. This library was used without modification. 

Lars Vogel's XML-reader written in Java has been used to read and store the routing-data downloaded from OpenStreetMap\cite{Vogella-XML}. Most of the code has been altered heavily to better fit my system and the data in the \textit{.osm} file downloaded from OpenStreetMap.\\
The tutorial is open content under the CC BY-NC-SA 3.0 DE license\cite{CC-License}.
The source-code is distributed under the Eclipse Public License\cite{Eclipse_License}.




Mapsforge's sample-code for displaying a map in Java has been used to display the map used by this system. Accessed 26.August.2016. The Mapsforge project is open source and is available on GitHub\cite{mapsforgeMap}.\\The sample-code is released under the GNU Lesser General Public License (GNU LGPL)\cite{GNU-LGPL}.