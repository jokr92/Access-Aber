\thispagestyle{empty}

\begin{center}
    {\LARGE\bf Abstract}
\end{center}

%Include an abstract for your project. This should be no more than 300 words.
%Stands alsone as a very short version of the dissertation
%The abstract should:
%	State the scope and principal objectives of the project
%	Describe the methods
%	Summarize the results
%	State the principal conclusions

%Thoughts: PGM1520 Abstract-assignment. split into distinct sections, but still contained within one paragraph:
%Handed in:	Background - why - detail
%Feedback:	detail - why

%The abstract loses clarity by trying to say too many things in each sentence.  A better way to cope with a tight word limit is to decide what are the most important things to say, and what can safely be left out. In this case, it would be better to start with what your own work offers, and then to say how how it is likely to affect people with reduced mobility.

%%%%%DRAFT%%%%%

 Route-planning applications are designed to help their users find good paths between two or more locations on a map. These systems are useful to a wide variety of people -- like tourists travelling to unfamiliar areas, or companies wanting to minimise fuel-costs and/or travel-time for their delivery vehicles. Route-planning can even be a useful tool on planes and boats, as it can help guide vessels into areas where the forces of nature give less resistance, or away from restricted/dangerous areas.

Conventional route-planning software is often aimed at one or more large groups of specific users -- like motorists (commercial and/or private), pedestrians, cyclists, etc. But very few of these systems are able to plan good -- or even practical -- routes for Persons with Reduced Mobility (\textit{PRM}), as these users are often simply grouped together with pedestrians or cyclists; with no special consideration taken with respect to their physical limitations.

Built-up areas pose particularly difficult environments to navigate for PRMs, as many commonly encountered obstacles like stairs, non-automatic doors, curbs, and steep slopes are effectively impassable for people with certain physical limitations. Without the aid of a route-planner, PRMs would need to rethink their planned route to a location on the fly whenever they encounter an obstacle, which can often prove quite frustrating, as an accessible path to where they want to go may not even exist.

The route-planning system described in this thesis tries to address the aforementioned issues by identifying and avoiding inaccessible areas, and using accessible buildings as shortcuts in an attempt to shorten the routes returned to its users. A number of different pathfinding algorithms have been tested, each of which have their own advantages and disadvantages.
