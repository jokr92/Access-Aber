\chapter{Testing}

%Detailed descriptions of every test case are definitely not what is required here. What is important is to show that you adopted a sensible strategy that was, in principle, capable of testing the system adequately even if you did not have the time to test the system fully.

%Have you tested your system on \textquoteleft real users\textquoteright ? For example, if your system is supposed to solve a problem for a business, then it would be appropriate to present your approach to involve the users in the testing process and to record the results that you obtained. Depending on the level of detail, it is likely that you would put any detailed results in an appendix.

%The following sections indicate some areas you might include. Other sections may be more appropriate to your project. 

\section{Overall Approach to Testing}
\textbf{NOTES: - DELETE THIS -}
\begin{itemize}
	\item Test Driven Development (\textbf{TDD}).
	\subitem Tests have been written for every class and every method within those classes.
	\subitem Tests were usually written before or right after a new method was implemented.
	\item Top-level test-class for running every test together.
\end{itemize}


\section{Automated Testing}
\textbf{NOTES: - DELETE THIS -}
\begin{itemize}
	\item JUnit-tests created for every class and method. Top-level test-class used to run every test automatically.
\end{itemize}


\section{Integration Testing}
\textbf{NOTES: - DELETE THIS -}
\begin{itemize}
	\item Is this where I talk about the custom graph made for testing optimality and completeness?
	\item Nodes not visited in the custom graph-illustrations are meant to represent loops and dead ends.
\end{itemize}

\begin{figure}
	\centering
	\caption[Inaccessible path in the custom graph]{Inaccessible path in the custom graph. This route passes through a Way with the tag: \textit{highway=steps}, and is therefore inaccessible. It is the shortest path in the custom graph, but should never be expanded by any algorithm.}
	\label{fig:customInaccessible}
	\frame{\includegraphics[keepaspectratio, width=\columnwidth]{Images/Custom_graph-Inaccessible}}
\end{figure}

\begin{figure}
	\centering
	\caption[Longest path in the custom graph]{The longest path in the custom graph. This route is the longest with respect to distance, but passes through fewer Nodes than Figure \ref{fig:customShortest}. Uninformed search-algorithms like BFS and DFS, and greedy algorithms like GBFS are likely to follow this path.}
	\label{fig:customLongest}
	\frame{\includegraphics[keepaspectratio, width=\columnwidth]{Images/Custom_graph-Longest}}
\end{figure}

\begin{figure}
	\centering
	\caption[Optimal path in the custom graph]{The optimal path in the custom graph. This route is the shortest with respect to distance, but passes through more Nodes than Figure \ref{fig:customLongest}. Informed search-algorithms like A* should always pass through here. Algorithms that follow this path can be considered \textit{optimal}}
	\label{fig:customShortest}
	\frame{\includegraphics[keepaspectratio, width=\columnwidth]{Images/Custom_graph-Shortest}}
\end{figure}


\section{User Testing}
\textbf{NOTES: - DELETE THIS -}
\begin{itemize}
	\item The system has not been tested together with any PRMs.
\end{itemize}